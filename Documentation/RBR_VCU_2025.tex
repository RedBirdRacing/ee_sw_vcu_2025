\documentclass[a4paper,12pt]{article}
\usepackage{fontspec}
\usepackage{hyperref}
\usepackage{listings}
\usepackage{xcolor}
\usepackage{geometry}
\geometry{margin=1in}

% Listings settings for code
\lstset{
    basicstyle=\ttfamily\footnotesize, % Shrink the font size to footnotesize
    keywordstyle=\color{blue}\bfseries,
    commentstyle=\color{gray},
    stringstyle=\color{red},
    numbers=left,
    numberstyle=\tiny,
    stepnumber=1,
    numbersep=5pt,
    frame=single,
    breaklines=true,
    showstringspaces=false,
    tabsize=4,
    captionpos=b,
}

% Title and author
\title{Red Bird Racing EVRT Vehicle Control Unit (VCU) (2025) \\ Project Documentation}
\author{Red Bird Racing EVRT}
\date{\today}

\begin{document}

\maketitle

\tableofcontents
\newpage

\section{Introduction}
This document provides an overview of the Red Bird Racing EVRT Vehicle Control Unit (VCU) (2025). The VCU firmware is designed to manage pedal input, CAN communication, and vehicle state transitions for our Formula Student electric race car.

\subsection{Project Structure}
The project is organized as follows:
\begin{lstlisting}[basicstyle=\ttfamily\small]
.
+-- include
|   +-- Debug.h
|   +-- pinMap.h
|   +-- README
+-- lib
|   +-- Pedal
|   |   +-- Pedal.cpp
|   |   +-- Pedal.h
|   |   +-- library.json
|   +-- Queue
|   |   +-- Queue.cpp
|   |   +-- Queue.h
|   +-- Signal_Processing
|   |   +-- Signal_Processing.cpp
|   |   +-- Signal_Processing.h
|   +-- README
+-- src
|   +-- main.cpp
+-- test
|   +-- README
+-- platformio.ini
+-- .vscode
    +-- launch.json
    +-- extensions.json
    +-- c_cpp_properties.json
\end{lstlisting}

\section{Setup and Tuning}
\begin{enumerate}
    \item Adjust pedal input constants in \texttt{Pedal.h}.
    \item Flash the VCU firmware. Ensure the car is jacked up and powered off during this process.
    \item Clear the area around the car, especially the rear.
    \item Test the minimum and maximum pedal input voltages and adjust the constants accordingly.
\end{enumerate}

\section{Debugging}
Debugging is performed using the serial monitor. Enable specific debug messages by setting flags in \texttt{Debug.h}. Note that enabling debugging may introduce delays due to the slow serial communication.

\section{Reverse Mode}
Reverse mode is implemented for testing purposes only and is prohibited in competition. The driver must hold the reverse button to engage reverse mode. Releasing the button places the car in neutral. The car would re-enter reverse mode if criteria are met; else forward mode is engaged if its criteria are met.

\textbf{Important Notes:}
{\fontspec{Microsoft YaHei}
\begin{itemize}
    \item \textbf{Do NOT use in actual competition!}
    \item \textbf{Rules 5.2.2.3: 禁止通过驱动装置反转车轮。}
    \item Rough translation: It is prohibited to use the motor to turn the wheels backwards.
\end{itemize}
}

\subsection{Reverse Mode Logic}
The reverse mode logic in \texttt{Pedal.cpp} allows the driver to toggle between reverse and forward modes using a single button. Below is the updated workflow and key components:

\subsubsection{Key Functions}
\begin{itemize}
    \item \texttt{void pedal\_can\_frame\_update(can\_frame *tx\_throttle\_msg, unsigned long millis)}:
    Updates the CAN frame with the current throttle value and handles reverse mode logic. The reverse button toggles between reverse and forward modes:
    \begin{itemize}
        \item If the reverse button is pressed, the mode toggles between reverse and forward.
        \item If \texttt{reverseMode} is \texttt{true}, the buzzer is activated, and reverse torque is calculated.
    \end{itemize}

    \item \texttt{int calculateReverseTorque(float throttleVolt, float vehicleSpeed, int torqueRequested)}:
    Calculates the torque value for reverse mode with the following constraints:
    \begin{itemize}
        \item The throttle voltage must be less than one-third of the maximum throttle voltage (\texttt{MAX\_THROTTLE\_IN\_VOLT / 3}).
        \item The vehicle speed must not exceed the reverse speed limit (\texttt{REVERSE\_SPEED\_MAX}).
        \item The torque is scaled down to 30\% of the requested torque to ensure reverse mode is slow and controllable.
    \end{itemize}
    If any of the constraints are violated, the torque is set to zero.
\end{itemize}

\subsubsection{Reverse Mode Workflow}
\begin{enumerate}
    \item The reverse button state is read using \texttt{digitalRead(reverse\_pin)}.
    \item If the reverse button is pressed, and conditions are met:
    \begin{itemize}
        \item If the vehicle is in forward mode (\texttt{reverseMode = false}), it switches to reverse mode (\texttt{reverseMode = true}).
        \item If the vehicle is in reverse mode (\texttt{reverseMode = true}), it switches to forward mode (\texttt{reverseMode = false}).
    \end{itemize}
    \item In reverse mode:
    \begin{itemize}
        \item A buzzer is activated with a periodic beep to alert nearby individuals. The cycle time is {BUZZER\_CYCLE\_TIME} milliseconds.
        \item The reverse torque is calculated using \texttt{calculateReverseTorque}.
    \end{itemize}
    \item The throttle torque value is updated and sent via CAN messages.
    \item If the motor direction needs to be flipped (e.g., for forward mode), the torque value is negated.
\end{enumerate}

\subsubsection{Safety Notes}
\begin{itemize}
    \item Reverse mode is implemented for testing purposes only and should not be used in competition.
    \item \textbf{Rules 5.2.2.3}
    \item Rough translation: It is prohibited to use the motor to turn the wheels backwards.
    \item The reverse mode logic ensures that the vehicle operates safely by limiting throttle and speed in reverse mode.
    \item However, care should be taken any time the vehicle is maneuvering, or if the buzzer is heard.
\end{itemize}

\section{Source Code Overview}
\subsection{\texttt{main.cpp}}
The main file initializes the pedal, CAN communication, and state machine for the car. It handles transitions between states such as \texttt{INIT}, \texttt{IN\_STARTING\_SEQUENCE}, \texttt{BUZZING}, and \texttt{DRIVE\_MODE}.

\lstinputlisting[language=C++,caption={\texttt{main.cpp}},label={lst:main}]{c:/Users/ckt63/Documents/GitHub/ee_sw_vcu_2025/src/main.cpp}

\subsection{\texttt{Pedal.cpp} and \texttt{Pedal.h}}
These files define the \texttt{Pedal} class, which encapsulates functionality for reading pedal input, filtering signals, and constructing CAN frames.

\lstinputlisting[language=C++,caption={\texttt{Pedal.cpp}},label={lst:pedalcpp}]{c:/Users/ckt63/Documents/GitHub/ee_sw_vcu_2025/lib/Pedal/Pedal.cpp}

\lstinputlisting[language=C++,caption={\texttt{Pedal.h}},label={lst:pedalh}]{c:/Users/ckt63/Documents/GitHub/ee_sw_vcu_2025/lib/Pedal/Pedal.h}

\subsection{\texttt{Queue.cpp} and \texttt{Queue.h}}
These files implement a static FIFO queue and a ring buffer for managing pedal input data.

\lstinputlisting[language=C++,caption={\texttt{Queue.cpp}},label={lst:queuecpp}]{c:/Users/ckt63/Documents/GitHub/ee_sw_vcu_2025/lib/Queue/Queue.cpp}

\lstinputlisting[language=C++,caption={\texttt{Queue.h}},label={lst:queueh}]{c:/Users/ckt63/Documents/GitHub/ee_sw_vcu_2025/lib/Queue/Queue.h}

\subsection{\texttt{Signal\_Processing.cpp} and \texttt{Signal\_Processing.h}}
These files provide simple DSP functions for filtering and processing pedal input signals.

\lstinputlisting[language=C++,caption={\texttt{Signal\_Processing.cpp}},label={lst:signalcpp}]{c:/Users/ckt63/Documents/GitHub/ee_sw_vcu_2025/lib/Signal_Processing/Signal_Processing.cpp}

\lstinputlisting[language=C++,caption={\texttt{Signal\_Processing.h}},label={lst:signalh}]{c:/Users/ckt63/Documents/GitHub/ee_sw_vcu_2025/lib/Signal_Processing/Signal_Processing.h}

\subsection{\texttt{Debug.h}}
This file defines macros for enabling or disabling debug messages.

\lstinputlisting[language=C++,caption={\texttt{Debug.h}},label={lst:debugh}]{c:/Users/ckt63/Documents/GitHub/ee_sw_vcu_2025/include/Debug.h}

\subsection{\texttt{pinMap.h}}
This file maps the pins used in the project to meaningful names.

\lstinputlisting[language=C++,caption={\texttt{pinMap.h}},label={lst:pinmaph}]{c:/Users/ckt63/Documents/GitHub/ee_sw_vcu_2025/include/pinMap.h}

\section{PlatformIO Configuration}
The \texttt{platformio.ini} file configures the PlatformIO environment for the project. It specifies the board, framework, and library dependencies.

\lstinputlisting[caption={\texttt{platformio.ini}},label={lst:platformio}]{c:/Users/ckt63/Documents/GitHub/ee_sw_vcu_2025/platformio.ini}

\section{Future Development}
\begin{itemize}
    \item Add more CAN channels for BMS, data logger, and other components.
    \item Improve the torque curve for better performance.
    \item Fully implement reverse mode.
\end{itemize}

\section{References}
\begin{itemize}
    \item \href{https://docs.platformio.org/en/latest/}{PlatformIO Documentation}
    \item \href{https://gcc.gnu.org/onlinedocs/cpp/Header-Files.html}{GCC Header File Documentation}
\end{itemize}

\end{document}